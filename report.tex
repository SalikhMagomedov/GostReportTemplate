\documentclass[utf8,usehyperref,12pt]{G7-32}
\usepackage{amsthm,amsfonts,amsmath,amssymb,amscd}
\usepackage[T2A]{fontenc}
\usepackage[utf8]{inputenc}
\usepackage[english,russian]{babel}
\usepackage{float}
\usepackage{graphicx}
\usepackage{cmap}
\usepackage{color}


\usepackage[all,cmtip]{xy}
\graphicspath{{pictures/}}

\usepackage{subcaption}
\usepackage{cite}

\usepackage{dsfont}
\usepackage{mathrsfs}
%\newtheorem{theoremA}{Теорема}
%\renewcommand*{\thetheoremA}{\Alph{theoremA}}

\TableInChaper
\PicInChaper
\setlength\GostItemGap{2mm}


\NirOrgLongName{\MakeUppercase{Название организации}}

\NirBoss{Должность руководителя организации}{ФИО} %% Заказчик, утверждающий НИР
\NirManager{Должность руководителя НИР}{ФИО}

\NirTown{[ГОРОД],}
\NirYear{[ГОД ОТЧЕТА]}

\NirUdk{УДК \No }
\NirGosNo{Регистрационный \No }

\NirStage{
}{[ТИП ОТЧЕТА], за [ГОД] г.}{
}

\bibliographystyle{unsrt}

\newtheorem{theorem}{Теорема}[chapter]
\newtheorem*{theorem*}{Теорема}
\newtheorem*{lemma*}{Лемма}
\newtheorem{property}{Свойство}
\newtheorem{lemma}{Лемма}[chapter]
\newtheorem{statement}{Утверждение}[chapter]
\newtheorem{definition}{Определение}[chapter]
\newtheorem{example}{Пример}[chapter]
\newtheorem{corollary}{Следствие}[chapter]
\newtheorem*{corollary*}{Следствие}
\newtheorem{remark}{Замечание}[chapter]
\newtheorem*{remark*}{Замечание}
\newtheorem{hypothesis}{Гипотеза}[chapter]

\newtheorem{cond}{Условие}
\newtheorem{theoremA}{Теорема}
\newtheorem{lemmaA}{Лемма}[section]
\newtheorem{state}{Предложение}
\newtheorem{proposition}{Предложение}
\renewcommand{\thetheoremA}{\Alph{theoremA}}
\renewcommand{\thelemmaA}{\thesection.\Alph{lemmaA}}
\newcommand{\No}{\textnumero}

\newcommand{\norm}[1]{\|#1\|_{p(\cdot),w}}
\newcommand{\ip}[2]{\langle #1, #2 \rangle}

\DeclareMathOperator*{\esssup}{ess\,sup}
\DeclareMathOperator*{\essinf}{ess\,inf}

\numberwithin{equation}{chapter} %
\renewcommand{\theequation}{\thechapter.\arabic{equation}}

\newenvironment{description}{}{}

\newcommand{\ifNotEmpty}[2] {
    \ifx&#1& \empty \else #2 \fi
}

\newcommand{\row}[3][]{
    \small{#2}&
    \centering{\rule[-2mm]{5cm}{0.2mm}} \newline \centering \footnotesize{{подпись, дата}} &
    #3 \newline \small{{\ifNotEmpty{#1}{(#1)}}} \\ & & \\
}

\usepackage[shortlabels]{enumitem}

%%%%%%%<------------- НАЧАЛО ДОКУМЕНТА
\begin{document}

% \usefont{T2A}{ftm}{m}{} %%% Использование шрифтов Т2 для возможности скопировать текст из PDF-файлов.

\frontmatter %%% <-- это выключает нумерацию ВСЕГО; здесь начинаются ненумерованные главы типа Исполнители, Обозначения и прочее

\NirTitle{\begin{center}
{\large
[НАЗВАНИЕ ТЕМЫ]
}
\\[12pt]
\end{center}
}

\Executors %% Список исполнителей здесь
% %%% это рисует линию размера 3мм и толщиной 0.1 пункт
\begin{longtable}{p{0.35\linewidth} p{0.305\linewidth} p{0.33\linewidth}}
    
    Руководитель НИР, & & \\
    \row[№ РАЗДЕЛА]{[ДОЛЖНОСТЬ]}{[И.О. ФАМИЛИЯ]}

    Отв. исполнители: & & \\
    \row[№ РАЗДЕЛА]{[ДОЛЖНОСТЬ]}{[И.О. ФАМИЛИЯ]}
    \row[№ РАЗДЕЛА]{[ДОЛЖНОСТЬ]}{[И.О. ФАМИЛИЯ]}
    \row[№ РАЗДЕЛА]{[ДОЛЖНОСТЬ]}{[И.О. ФАМИЛИЯ]}

    Исполнители: & & \\
    \row[№ РАЗДЕЛА]{[ДОЛЖНОСТЬ]}{[И.О. ФАМИЛИЯ]}
    \row[№ РАЗДЕЛА]{[ДОЛЖНОСТЬ]}{[И.О. ФАМИЛИЯ]}
    \row[№ РАЗДЕЛА]{[ДОЛЖНОСТЬ]}{[И.О. ФАМИЛИЯ]}

    \row[]{Нормоконтроль}{[ФИО]}

\end{longtable}


% Настройки реферата
% [Количество книг]
% \TotalBooks{0}
% [Количество рисунков]
% \TotalFigures{0}
% [Количество таблиц]
% \TotalTables{0}
% [Количество использованных источников]
% \TotalBibItems{0}
% [Количество приложения]
% \TotalAppendixes{0}

% Ключевые слова
\KeyWords{Ключевые слова}

% Оптимальный объем (~850 печатных знаков)
[Реферат отчёта, не более 1 страницы]

\setcounter{tocdepth}{2} %hide subsections

\tableofcontents

%\NormRefs % Нормативные ссылки
%\Defines % Необходимые определения

\Abbreviations %% Список обозначений и сокращений в тексте

[В этом файле должно быть записано общее для отдела введение]

\mainmatter %% это включает нумерацию глав и секций в документе ниже

\chapter{Тема основной части отчета сотрудника}

[Здесь должны содержаться основные результаты отчета с учетом введения и заключения]

\backmatter %% Здесь заканчивается нумерованная часть документа и начинаются заключение и ссылки

\Conclusion

[Общее заключение для всей темы отчета]% заключение к отчёту
\begin{thebibliography}{111}

  % Ниже указаны примеры форматирования литературы

  \bibitem{mmg-MarcellanXu2015}
  Marcellán F., Xu Y. On Sobolev orthogonal polynomials // Expositiones Math. --- 2015. --- Vol 33. P. 308---352.

  \bibitem{mmg-mmg-walsh-Shii-UMN}
  Шарапудинов И.И. Ортогональные по Соболеву системы функций и некоторые их приложения // УМН. --- 2019. --- Т. 74, \No 4(448). --- С. 87---164.

  \bibitem{ark-bib-2}
  Стечкин С.Б., Субботин Ю.Н.
  Сплайны в вычислительной математике.
  --- М.: Наука,
  1976. --- 248~с.

  \bibitem{ark-bib-3}
  Завьялов Ю.С., Квасов Б.И., Мирошниченко В.Л.
  Методы сплайн-функций.
  --- М.: Наука,
  1980.
  --- 352 c.

\end{thebibliography} 
\chapter{Cписок работ, опубликованных \texorpdfstring{\\ }{} по теме НИР в [ГОД] г.}

% Ниже представлены примеры опубликованных работ

\section*{Список опубликованных научных статей}

\begin{enumerate}
    \item
    {[Авторы]}
    [Название статьи]
    //
    [Название журнала].
    --- [год выпуска].
    --- Т. [номер тома], вып [номер выпуска].
    --- С.~[страница начала статьи]---[страница конца статьи].

    \item
    \foreignlanguage{english}{%
        Sultanakhmedov, M.S.
        Approximation of Functions by Discrete Fourier Sums in Polynomials Orthogonal on a Nonuniform Grid with Jacobi Weight.
        //
        Math Notes. 
        --- 2021.
        --- Vol. 110.
        --- P.~418---431.
    }%

    \item
    \foreignlanguage{english}{%
        Gadzhimirzaev, R.M., Shakh-Emirov, T.N.
        Approximation Properties of the Vallée-Pous\-sin Means of Partial Sums of a Special Series in Laguerre Polynomials.
        //
        Math Notes. 
        --- 2021.
        --- Vol. 110.
        --- P.~475---488.
    }%

\end{enumerate}

\section*{Список зарегистрированных программ для ЭВМ}

\begin{enumerate}

    \item
    {[Авторы]} Свидетельство №[номер свидетельства] о государственной регистрации программы для ЭВМ <<[название программы]>>. Заявка №[номер заявки], дата поступления [дата поступления заявки]. Дата государственной регистрации в Реестре программ для ЭВМ [дата гос. регистрации]. Правообладатель: ДФИЦ РАН.

    \item
    {[Авторы]} Свидетельство №[номер свидетельства] о государственной регистрации программы для ЭВМ <<[название программы]>>. Заявка №[номер заявки], дата поступления [дата поступления заявки]. Дата государственной регистрации в Реестре программ для ЭВМ [дата гос. регистрации]. Правообладатель: ДФИЦ РАН.

\end{enumerate}


\end{document}
